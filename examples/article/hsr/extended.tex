%% ----------------------------------------------------------------
%% Thesis.tex -- MAIN FILE (the one that you compile with LaTeX)
%% ----------------------------------------------------------------

% Set up the document
\documentclass[a4paper, 11pt, oneside]{article/HSR}  % Use the "Thesis" style, based on the ECS Thesis style by Steve Gunn
\graphicspath{Figures/}  % Location of the graphics files (set up for graphics to be in PDF format)
\usepackage[english]{babel}

\usepackage{multicol}

% Include any extra LaTeX packages required
\usepackage[square, numbers, comma, sort&compress]{natbib}  % Use the "Natbib" style for the references in the Bibliography
\usepackage{verbatim}  % Needed for the "comment" environment to make LaTeX comments
\hypersetup{urlcolor=blue, colorlinks=true}  % Colours hyperlinks in blue, but this can be distracting if there are many links.
\usepackage{blindtext} % for example text. TODO remove for a real document

%% ----------------------------------------------------------------
\begin{document}
  % Set up the Title Page
  \title  {Thesis Title}
  \authors   {\texorpdfstring{\href{hans.muster@test.ch}{Author Name}}{Author Name}}
  \addresses {\groupname\\\deptname\\\univname}  % Do not change this here, instead these must be set in the "Thesis.cls" file, please look through it instead
  \date      {\today}
  \subject   {}
  \keywords  {}

  \maketitle
  %% ----------------------------------------------------------------

  \setstretch{1.3}  % It is better to have smaller font and larger line spacing than the other way round

  \begin{abstract}
    The Thesis Abstract is written here (and usually kept to just this page). The page is kept centered vertically so can expand into the blank space above the title too\ldots
  \end{abstract}

  \clearpage

  \tableofcontentspage

  % Include the sections of the article, as separate files
  % Just uncomment the lines as you write the sections

  %\input{Sections/Section1} % Introduction

  %\input{Sections/Section2} % Background Theory

  %\input{Sections/Section3} % Experimental Setup

  %\input{Sections/Section4} % Experiment 1

  %\input{Sections/Section5} % Experiment 2

  %\input{Sections/Section6} % Results and Discussion

  %\input{Sections/Section7} % Conclusion

  \begin{multicols}{2}[\section{Introduction}]
    \label{introduction}
    This is time for all good men to come to the aid of their party!

    \subsection{Test Section}\label{test section}
    \Blindtext
    \Blindtext
    \Blindtext

    \subsubsection{Previous work}\label{previous work}
    You might want to read more from \citet{einstein}.
    \vfill\null
    \clearpage
  \end{multicols}

  \begin{multicols}{2}[\section{Results}]
    \label{results}
    In this section we describe the results.
    \Blindtext
  \end{multicols}

  \begin{multicols}{2}[\section{Conclusions}]
    \label{conclusions}
    We worked hard, and achieved very little.
    \Blindtext
  \end{multicols}

  %% ----------------------------------------------------------------
  % Now begin the Appendices, including them as separate files

  \addtocontents{toc}{\vspace{2em}} % Add a gap in the Contents, for aesthetics

  \appendix % Cue to tell LaTeX that the following 'sections' are Appendices

  %\input{Appendices/AppendixA}  % Appendix Title

  %\input{Appendices/AppendixB} % Appendix Title

  %\input{Appendices/AppendixC} % Appendix Title

  \begin{multicols}{2}[\section{An Appendix}]
    \label{an_appendix}
    \Blindtext
  \end{multicols}

  \addtocontents{toc}{\vspace{2em}}  % Add a gap in the Contents, for aesthetics

  %% ----------------------------------------------------------------

  %\label{bibliography}
  \bibliographystyle{unsrtnat}  % Use the "unsrtnat" BibTeX style for formatting the Bibliography
  \bibliography{sample}  % The references (bibliography) information are stored in the file named "sample.bib"
\end{document}
