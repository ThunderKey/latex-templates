\newcommand\bhrule{\typeout{------------------------------------------------------------------------------}}
\newcommand\btypeout[1]{\bhrule\typeout{\space #1}\bhrule}

\newcommand\Declaration[1]{
  \btypeout{Declaration of Authorship}
  \addtotoc{Declaration of Authorship}
  \thispagestyle{plain}
  \null\vfil
  %\vskip 60\p@
  \begin{center}{\huge\bf Declaration of Authorship\par}\end{center}
  %\vskip 60\p@
  {\normalsize #1}
  \vfil\vfil\null
  %\cleardoublepage
}

\newcommand{\declarationpage}[0]{
  % Declaration Page required for the Thesis, your institution may give you a different text to place here
  \Declaration{
    \addtocontents{toc}{\vspace{1em}}  % Add a gap in the Contents, for aesthetics

    I, \authornames, declare that this thesis titled, `\@title' and the work presented in it are my own. I confirm that:

    \begin{itemize}
      \renewcommand\labelitemi{\tiny{$\blacksquare$}}
      \item This work was done wholly or mainly while in candidature for a research degree at this University.
      \item Where any part of this thesis has previously been submitted for a degree or any other qualification at this University or any other institution, this has been clearly stated.
      \item Where I have consulted the published work of others, this is always clearly attributed.
      \item Where I have quoted from the work of others, the source is always given. With the exception of such quotations, this thesis is entirely my own work.
      \item I have acknowledged all main sources of help.
      \item Where the thesis is based on work done by myself jointly with others, I have made clear exactly what was done by others and what I have contributed myself.
    \end{itemize}

    Signed:\\
    \rule[1em]{25em}{0.5pt}  % This prints a line for the signature

    Date:\\
    \rule[1em]{25em}{0.5pt}  % This prints a line to write the date
  }
  \clearpage  % Declaration ended, now start a new page
}
\newcommand{\quotepage}[2]{
  \thispagestyle{empty}  % No headers or footers for the following pages

  \null\vfill
  % Now comes the "Funny Quote", written in italics
  \textit{``#1''}

  \begin{flushright}
    #2
  \end{flushright}

  \vfill\vfill\vfill\vfill\vfill\vfill\null
  \clearpage  % Funny Quote page ended, start a new page
}
\newcommand{\abstractpage}[1]{
  % The Abstract Page
  \addtotoc{Abstract}  % Add the "Abstract" page entry to the Contents
  \abstract{
    \addtocontents{toc}{\vspace{1em}}  % Add a gap in the Contents, for aesthetics
    #1
  }

  \clearpage  % Abstract ended, start a new page
}
\newcommand{\acknowledgementspage}[1]{
  % The Acknowledgements page, for thanking everyone
  \acknowledgements{
    \addtocontents{toc}{\vspace{1em}}  % Add a gap in the Contents, for aesthetics
    #1
  }
  \cleardoublepage  % End of the Acknowledgements
}
\newcommand{\tableofcontentspage}[0]{
  \tableofcontents  % Write out the Table of Contents
  \cleardoublepage
}
\newcommand{\listoffigurespage}[0]{
  \listoffigures  % Write out the List of Figures
  \cleardoublepage
}
\newcommand{\listoftablespage}[0]{
  \listoftables  % Write out the List of Tables
  \cleardoublepage
}
\newcommand{\listofabbreviationspage}[1]{
  \listofabbreviations
  \cleardoublepage  % Start a new page
}
\newcommand{\listofabbreviations}[1]{
  \setstretch{1.5}  % Set the line spacing to 1.5, this makes the following tables easier to read

  \listofsymbols{ll}{ % Include a list of Abbreviations (a table of two columns)
    #1
  }

  \setstretch{1.3}  % Return the line spacing back to 1.3
}
\newcommand{\symbolspage}[1]{
  \setstretch{1.5}  % Set the line spacing to 1.5, this makes the following tables easier to read

  \listofnomenclature{lll}{  % Include a list of Symbols (a three column table)
    #1
  }

  \setstretch{1.3}  % Return the line spacing back to 1.3
}
\newcommand{\dedicationpage}[1]{
  % End of the pre-able, contents and lists of things
  % Begin the Dedication page

  \thispagestyle{empty}  % Page style needs to be empty for this page
  \dedicatory{#1}

  \addtocontents{toc}{\vspace{2em}}  % Add a gap in the Contents, for aesthetics
}

\newcommand{\chapterorsection}[1]{
  \@ifundefined{chapter}{
    \section*{#1}
  }{
    \chapter*{#1}
  }
}
\addtocounter{secnumdepth}{1}
\setcounter{tocdepth}{6}
\newcounter{dummy}
\newcommand\addtotoc[1]{
  \refstepcounter{dummy}
  \@ifundefined{chapter}{
    \addcontentsline{toc}{section}{#1}
  }{
    \addcontentsline{toc}{chapter}{#1}
  }
}
\renewcommand\tableofcontents{
  \btypeout{Table of Contents}
  \begin{spacing}{1}{
    \setlength{\parskip}{1pt}
    \if@twocolumn
      \@restonecoltrue\onecolumn
    \else
      \@restonecolfalse
    \fi
    \chapterorsection{
      \contentsname
      \@mkboth{\MakeUppercase\contentsname}{\MakeUppercase\contentsname}
    }
    \@starttoc{toc}
    \if@restonecol\twocolumn\fi
  }\end{spacing}
}
\renewcommand\listoffigures{
  \btypeout{List of Figures}
  \addtotoc{List of Figures}
  \begin{spacing}{1}{
    \setlength{\parskip}{1pt}
    \if@twocolumn
      \@restonecoltrue\onecolumn
    \else
      \@restonecolfalse
    \fi
    \chapterorsection{\listfigurename
      \@mkboth{\MakeUppercase\listfigurename}
              {\MakeUppercase\listfigurename}}
    \@starttoc{lof}
    \if@restonecol\twocolumn\fi
  }\end{spacing}
}
\renewcommand\listoftables{
  \btypeout{List of Tables}
  \addtotoc{List of Tables}
  \begin{spacing}{1}{
    \setlength{\parskip}{1pt}
    \if@twocolumn
      \@restonecoltrue\onecolumn
    \else
      \@restonecolfalse
    \fi
    \chapterorsection{\listtablename
      \@mkboth{
        \MakeUppercase\listtablename}{\MakeUppercase\listtablename}}
      \@starttoc{lot}
    \if@restonecol\twocolumn\fi
  }\end{spacing}
}
\newcommand\listsymbolname{Abbreviations}
\usepackage{longtable}
\newcommand\listofsymbols[2]{
  \btypeout{\listsymbolname}
  \addtotoc{\listsymbolname}
  \chapterorsection{
    \listsymbolname
    \@mkboth{\MakeUppercase\listsymbolname}{\MakeUppercase\listsymbolname}
  }
  \begin{longtable}[c]{#1}#2\end{longtable}\par
  \cleardoublepage
}
\newcommand\listconstants{Physical Constants}
\usepackage{longtable}
\newcommand\listofconstants[2]{
  \btypeout{\listconstants}
  \addtotoc{\listconstants}
  \chapterorsection{
    \listconstants
    \@mkboth{\MakeUppercase\listconstants}{\MakeUppercase\listconstants}
  }
  \begin{longtable}[c]{#1}#2\end{longtable}\par
  \cleardoublepage
}
\newcommand\listnomenclature{Symbols}
\usepackage{longtable}
\newcommand\listofnomenclature[2]{
  \btypeout{\listnomenclature}
  \addtotoc{\listnomenclature}
  \chapterorsection{
    \listnomenclature
    \@mkboth{\MakeUppercase\listnomenclature}{\MakeUppercase\listnomenclature}
  }
  \begin{longtable}[c]{#1}#2\end{longtable}
  \par
  \cleardoublepage
}

\newcommand\acknowledgementstitle{Acknowledgements}
\newcommand\acknowledgements[1]{
  \btypeout{\acknowledgementstitle}
  \addtotoc{\acknowledgementstitle}
  \chapterorsection{
    \begin{center}
      \acknowledgementstitle
      \@mkboth{\MakeUppercase\acknowledgementstitle}{\MakeUppercase\acknowledgementstitle}
    \end{center}
  }
  #1
  \cleardoublepage
}
\newcommand\dedicatory[1]{
  \btypeout{Dedicatory}
  \thispagestyle{plain}
  \null\vfil
  \vskip 60\p@
  \begin{center}{\Large \sl #1}\end{center}
    \vfil\null
  \cleardoublepage
}
